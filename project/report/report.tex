\documentclass[a4paper,12pt]{article}

\usepackage{fontspec}
\usepackage{microtype}
\usepackage{polyglossia}

\setmainlanguage{greek}
\setotherlanguage{english}

\setmainfont{Latin Modern Roman}
\newfontfamily\greekfont{CMU Serif}[Script=Greek]
\newfontfamily\greekfonttt{CMU Typewriter Text}[Script=Greek]

\usepackage[a4paper, top=1.5cm, bottom=1.5cm, left=1.5cm, right=1.5cm]{geometry}
\usepackage{graphicx}
\usepackage{float}
\usepackage{array}
\usepackage{hyperref}
\hypersetup{
    colorlinks=true,
    linkcolor=black,
    citecolor=black,
    urlcolor=blue
}
\usepackage[super,sort&compress]{cite}
\usepackage{amsmath}
\usepackage{algorithm}
\usepackage{algpseudocode}

\title{Τεχνικές Βελτιστοποίησης: Project}
\author{Ιωάννης Μιχάλαινας ΑΕΜ:10902}
\date{Δεκέμβριος 2025}

\begin{document}

\maketitle

\begin{center}
\textbf{Αποθετήριο Κώδικα:} \\
\href{https://github.com/ioannisam/optimization-techniques}
     {github.com/ioannisam/optimization-techniques}
\end{center}

\bigskip

\begin{abstract}
    Η παρούσα εργασία πραγματεύεται την προσέγγιση μιας άγνωστης μη γραμμικής συνάρτησης δύο μεταβλητών, $f(u_1, u_2)$, μέσω ενός Γενετικού Αλγορίθμου. Στόχος είναι η εύρεση μιας αναλυτικής έκφρασης που αποτελείται από γραμμικό συνδυασμό Γκαουσιανών συναρτήσεων. Ο αλγόριθμος υλοποιήθηκε εξ ολοκλήρου στο MATLAB, χωρίς τη χρήση έτοιμων εργαλείων βελτιστοποίησης. Η διαδικασία περιλαμβάνει την εκπαίδευση του μοντέλου, την επαλήθευση σε νέα δεδομένα για την αποφυγή υπερπροσαρμογής και τέλος, μια διαδικασία απλοποίησης για τη μείωση της πολυπλοκότητας της τελικής έκφρασης.
\end{abstract}

\tableofcontents
\clearpage

\section{Εισαγωγή}
Στο πλαίσιο της παρούσας εργασίας, εξετάζεται το πρόβλημα της εκτίμησης μιας άγνωστης, μη γραμμικής συνάρτησης δύο μεταβλητών $y = f(u_1, u_2)$, βάσει μετρήσεων εισόδου-εξόδου. Η αναλυτική μορφή της $f$ θεωρείται άγνωστη, ενώ είναι γνωστό ότι πρόκειται για συνεχή συνάρτηση των μεταβλητών $u_1$ και $u_2$.

Για την προσέγγιση της $f$, χρησιμοποιείται ένα παραμετρικό μοντέλο που βασίζεται σε γραμμικό συνδυασμό Γκαουσιανών συναρτήσεων βάσης. Συγκεκριμένα, η προσεγγιστική συνάρτηση ορίζεται ως:
\begin{equation}
    \hat f(u_1,u_2) = \sum_{k=1}^{K} G_k(u_1,u_2),
\end{equation}
όπου $K = 15$ και κάθε όρος $G_k$ έχει τη μορφή:
\begin{equation}
    G(u_1, u_2) = w \cdot e^{-\left(\frac{(u_1-c_1)^2}{2\sigma_1^2} + \frac{(u_2-c_2)^2}{2\sigma_2^2}\right)}.
\end{equation}

Η παράμετρος $w$ αποτελεί τον συντελεστή στάθμισης του κάθε Γκαουσιανού όρου, ενώ τα $c_1, c_2$ και $\sigma_1, \sigma_2$ καθορίζουν τη θέση και τη διασπορά του αντίστοιχα. Για κάθε Γκαουσιανή συνάρτηση απαιτείται ο προσδιορισμός πέντε παραμέτρων, με αποτέλεσμα το συνολικό διάνυσμα παραμέτρων να έχει διάσταση $5K = 75$.

Η διαδικασία εκπαίδευσης του μοντέλου πραγματοποιείται μέσω μετρήσεων εισόδου-εξόδου. Για τις ανάγκες παραγωγής των δεδομένων εκπαίδευσης και επαλήθευσης (διακριτά σύνολα), χρησιμοποιείται η συνάρτηση:
\begin{equation}
    f(u_1,u_2) = \sin(u_1 + u_2)\sin(u_2^2),
\end{equation}
με:
\[
    u_1 \in [-1,2], \quad u_2 \in [-2,1].
\]

Στα \textit{Σχήματα \ref{fig:visualization1} και \ref{fig:visualization2}} οπτικοποιείται η συνάρτηση $f(u_1,u_2)$ για διαφορετικό πλήθος δειγμάτων. Παρατηρείται ότι με την αύξηση του αριθμού των σημείων εισόδου-εξόδου καθίσταται σαφέστερη η μορφολογία της συνάρτησης.

\begin{figure}[H]
    \centering
    \begin{minipage}{0.48\linewidth}
        \centering
        \includegraphics[width=\linewidth]{assets/main_1.jpg}
        \caption{Οπτικοποίηση της $f(u_1,u_2)$ \\ (100 δείγματα)}
        \label{fig:visualization1}
    \end{minipage}
    \hfill
    \begin{minipage}{0.48\linewidth}
        \centering
        \includegraphics[width=\linewidth]{assets/main_2.jpg}
        \caption{Οπτικοποίηση της $f(u_1,u_2)$ \\ (1000 δείγματα)}
        \label{fig:visualization2}
    \end{minipage}
\end{figure}

Λόγω της έντονα μη γραμμικής φύσης του προβλήματος και του μεγάλου αριθμού προς εκτίμηση παραμέτρων, η εύρεση του βέλτιστου διανύσματος παραμέτρων $(w, c_1, \sigma_1, c_2, \sigma_2)$ πραγματοποιείται μέσω ενός Γενετικού Αλγορίθμου, ο οποίος παρουσιάζεται και αναλύεται στην επόμενη ενότητα.

\newpage
\section{Γενετικός Αλγόριθμος}
Οι Γενετικοί Αλγόριθμοι αποτελούν τεχνικές ολικής βελτιστοποίησης. Είναι ευριστικές μέθοδοι βελτιστοποίησης που εμπνέονται από τη διαδικασία της φυσικής επιλογής, η οποία έχει επιδείξει την επιτυχημένη επίλυση πολύπλοκων προβλημάτων βελτιστοποίησης, όπως η δημιουργία και ανάπτυξη νέων οργανισμών. 

\subsection{Περιγραφή}
Οι Γενετικοί Αλγόριθμοι λειτουργούν πάνω σε έναν πληθυσμό υποψήφιων λύσεων (άτομα), οι οποίες εξελίσσονται μέσω γενετικών τελεστών (επιλογή, διασταύρωση, μετάλλαξη) με στόχο την εύρεση της βέλτιστης λύσης σε πολύπλοκα προβλήματα όπου οι αναλυτικές μέθοδοι είναι δύσκολο να εφαρμοστούν.

Όλοι οι εξελικτικοί αλγόριθμοι θεωρούν έναν πληθυσμό από άτομα (χρωμοσώματα), καθένα από τα οποία αντιπροσωπεύει μία λύση του προβλήματος βελτιστοποίησης. Ο πληθυσμός εξελίσσεται σε γενεές, ως αποτέλεσμα της εφαρμογής διαφόρων εξελικτικών τελεστών. Με αυτό τον τρόπο δημιουργούνται νέα άτομα που παριστάνουν διαφορετικά σημεία στον χώρο αναζήτησης, καλύπτοντας ολοένα και περισσότερες περιοχές αυτού.

Σε κάθε άτομο του πληθυσμού αντιστοιχεί μια τιμή της αντικειμενικής συνάρτησης, η οποία στην ορολογία των εξελικτικών αλγορίθμων καλείται \textit{συνάρτηση ικανότητας} (fitness function). Η διαδικασία επιλογής διασφαλίζει ότι τα άτομα με την καλύτερη απόδοση, η οποία αξιολογείται με βάση τη συνάρτηση ικανότητας, θα έχουν μεγαλύτερη πιθανότητα να επιβιώσουν και να αναπαραχθούν. Κατά συνέπεια, ο πληθυσμός τείνει να εξελιχθεί προς λύσεις με βέλτιστη απόδοση.

Η δημιουργία νέων λύσεων (απογόνων) επιτυγχάνεται κυρίως μέσω δύο μηχανισμών:

\textbf{1. Διασταύρωση:}
Αποτελεί τον κύριο μηχανισμό ανταλλαγής πληροφορίας. Στην παρούσα εργασία, εφαρμόζεται η \textit{Αριθμητική Διασταύρωση}. Δοθέντων δύο γονέων $\mathbf{p}_1$ και $\mathbf{p}_2$, παράγονται δύο απόγονοι $\mathbf{c}_1, \mathbf{c}_2$ μέσω γραμμικού συνδυασμού:
\begin{equation}
    \begin{aligned}
        \mathbf{c}_1 &= \alpha \cdot \mathbf{p}_1 + (1-\alpha) \cdot \mathbf{p}_2 \\
        \mathbf{c}_2 &= (1-\alpha) \cdot \mathbf{p}_1 + \alpha \cdot \mathbf{p}_2
    \end{aligned}
\end{equation}
όπου $\alpha$ είναι ένας τυχαίος αριθμός στο διάστημα $(0,1)$. Η μέθοδος αυτή εξασφαλίζει ότι οι απόγονοι θα βρίσκονται εντός του υπερ-ορθογωνίου που ορίζουν οι γονείς.

\textbf{2. Μετάλλαξη:}
Η μετάλλαξη εισάγει τυχαίες μικρές διαταραχές στα γονίδια των απογόνων, αποτρέποντας την πρόωρη σύγκλιση σε τοπικά ακρότατα. Στην παρούσα υλοποίηση χρησιμοποιείται προσθετικός Γκαουσιανός θόρυβος:
\begin{equation}
    x'_{i} = x_{i} + \delta, \quad \text{όπου } \delta \sim \mathcal{N}(0, \sigma^2)
\end{equation}
όπου $x_i$ η τιμή του γονιδίου και $\delta$ ο θόρυβος που προκύπτει από την κανονική κατανομή με διασπορά που καθορίζεται από την παράμετρο θορύβου (\texttt{mutation\_noise}).

\begin{figure}[H]
    \centering
    \includegraphics[width=0.4\linewidth]{assets/ga.png}
    \caption{Εικονογράφηση των βημάτων ενός Γενετικού Αλγορίθμου\cite{gendesign}}
    \label{fig:ga}
\end{figure}

\subsection{Αλγόριθμος}
Εδώ παρουσιάζεται η παραπάνω διαδικασία του αλγοριθμικά σε ψευδοκώδικα και διαγραμματικά (\textit{Σχήμα \ref{fig:flowchart}}).

\begin{algorithm}[H]
    \caption{Γενετικός Αλγόριθμος Βελτιστοποίησης Παραμέτρων}
    \begin{algorithmic}[1]
        \Require 
        Μέγεθος πληθυσμού $N$, αριθμός Γκαουσιανών $K$, μέγιστος αριθμός γενεών $MaxGen$, πιθανότητα διασταύρωσης $p_c$, πιθανότητα μετάλλαξης $p_m$
        \Ensure Βέλτιστο διάνυσμα παραμέτρων $\theta_{best}$
        
        \State \textbf{Αρχικοποίηση:} Δημιουργία τυχαίου πληθυσμού $P_0$ μεγέθους $N$ εντός των επιτρεπτών ορίων.
        \State \textbf{Αξιολόγηση:} Υπολογισμός $MSE$ και Fitness $f(x) = \frac{1}{1+MSE(x)}$ για κάθε άτομο $x \in P_0$.
        
        \For{$gen = 1$ \textbf{to} $MaxGen$}
            \State \textbf{Ελιτισμός:} Εντοπισμός και αποθήκευση του καλύτερου ατόμου $x_{best}$ της τρέχουσας γενιάς.
            
            \State \textbf{Επιλογή:} Επιλογή $N$ γονέων από τον $P_{gen-1}$ με τη μέθοδο του Τροχού της Τύχης.
            
            \State \textbf{Διασταύρωση:} Εφαρμογή Διασταύρωσης σε ζεύγη γονέων με πιθανότητα $p_c$.
            
            \State \textbf{Μετάλλαξη:} Εφαρμογή θορύβου στα γονίδια των απογόνων με πιθανότητα $p_m$.
            
            \State \textbf{Επαναφορά Ορίων:} Διόρθωση τιμών που υπερβαίνουν τα όρια αναζήτησης.
            
            \State $P_{gen} \leftarrow$ Απόγονοι.
            \State \textbf{Εισαγωγή Elite:} Αντικατάσταση του πρώτου ατόμου του $P_{gen}$ με το $x_{best}$.
            
            \State \textbf{Αξιολόγηση:} Υπολογισμός Fitness για τον νέο πληθυσμό $P_{gen}$.
        \EndFor
        
        \State \Return Το άτομο με το μέγιστο Fitness από την τελευταία γενιά.
    \end{algorithmic}
\end{algorithm}

\begin{figure}[H]
    \centering
    \includegraphics[width=1\linewidth]{assets/flowchart.png}
    \caption{Διαγραμματική παρουσίαση του αλγορίθμου\cite{mathworks}}
    \label{fig:flowchart}
\end{figure}

\subsection{Παράμετροι}
Κατά την εκπόνηση της εργασίας ορίστηκαν ορισμένες παράμετροι και υπερπαράμετροι, η επιλογή των οποίων είναι ιδιαίτερης σημασίας, καθώς επηρεάζουν καθοριστικά τη συμπεριφορά και την απόδοση του αλγορίθμου. Η κατάλληλη ρύθμισή τους επιτρέπει την επίτευξη ικανοποιητικής ισορροπίας μεταξύ ακρίβειας και υπολογιστικής αποδοτικότητας. Στον \textot{Πίνακα \ref{tab:parameters}} που ακολουθεί συνοψίζονται οι επιλεγμένες τιμές των παραμέτρων, καθώς και το σκεπτικό που οδήγησε στην επιλογή τους για την παρούσα εργασία.

\begin{table}[H]
    \centering
    \begin{tabular}{|l|c|p{7cm}|}
    \hline
    \textbf{Παράμετρος} & \textbf{Τιμή} & \textbf{Περιγραφή / Αιτιολόγηση} \\ \hline
    \texttt{num\_samples} & 1000 & Πλήθος δειγμάτων εκπαίδευσης για επαρκή κάλυψη του πεδίου ορισμού. \\ \hline
    \texttt{c1\_range} & $[-1, 2]$ & Όρια αναζήτησης για τα κέντρα $c_1$, σύμφωνα με το πεδίο ορισμού του $u_1$. \\ \hline
    \texttt{c2\_range} & $[-2, 1]$ & Όρια αναζήτησης για τα κέντρα $c_2$, σύμφωνα με το πεδίο ορισμού του $u_2$. \\ \hline
    \texttt{w\_range} & $[-10, 10]$ & Περιορισμός του εύρους των βαρών για αποφυγή ακραίων τιμών. \\ \hline
    \texttt{sigma\_range} & $[0.1, 3.0]$ & Όρια για το πλάτος των Γκαουσιανών, ώστε να μην είναι ούτε πολύ "αιχμηρές" ούτε πολύ επίπεδες. \\ \hline
    \texttt{pop\_size} & 100 & Μέγεθος πληθυσμού που προσφέρει ισορροπία μεταξύ υπολογιστικού κόστους και ποικιλομορφίας. \\ \hline
    \texttt{num\_gaussians} & 15 & Ο αριθμός των Γκαουσιανών όρων, όπως ορίστηκε στις προδιαγραφές του project. \\ \hline
    \texttt{max\_generations} & 3000 & Μέγιστος αριθμός γενεών για να διασφαλιστεί η σύγκλιση του αλγορίθμου. \\ \hline
    \texttt{crossover\_rate} & 0.8 & Υψηλή πιθανότητα διασταύρωσης για αποτελεσματική εξερεύνηση του χώρου λύσεων. \\ \hline
    \texttt{mutation\_rate} & 0.05 & Χαμηλή πιθανότητα μετάλλαξης για διατήρηση της ποικιλομορφίας χωρίς να καταστρέφονται καλές λύσεις. \\ \hline
    \texttt{mutation\_noise} & 0.1 & Ένταση του θορύβου που προστίθεται κατά τη μετάλλαξη για μικρο-διορθώσεις. \\ \hline
    \texttt{pruning\_threshold} & 0.1 & Κατώφλι για την αφαίρεση (pruning) όρων με αμελητέα συνεισφορά στο τελικό μοντέλο. \\ \hline
    \end{tabular}
    \caption{Παράμετροι Γενετικού Αλγορίθμου}
    \label{tab:parameters}
\end{table}                                                 

\subsection{Αρχικοποίηση}
Η διαδικασία ξεκινά με τη δημιουργία ενός αρχικού πληθυσμού $N$ ατόμων (τυπικά 100 άτομα είναι ικανοποιητικά)\cite{book}. Στη συνάρτηση \texttt{initialize\_population} του κώδικα, κάθε γονίδιο του χρωμοσώματος λαμβάνει μια τυχαία τιμή από μια ομοιόμορφη κατανομή εντός των προκαθορισμένων ορίων (\textit{Πίνακας \ref{tab:parameters}}). Αυτό εξασφαλίζει ότι ο αλγόριθμος ξεκινά με μια ευρεία κάλυψη του χώρου αναζήτησης πριν αρχίσει η διαδικασία της εξέλιξης.

\subsection{Κωδικοποίηση}
Σε αντίθεση με τους κλασικούς γενετικούς αλγορίθμους που χρησιμοποιούν δυαδική κωδικοποίηση \cite{paper}, στην παρούσα εργασία επιλέχθηκε η κωδικοποίηση πραγματικών αριθμών (real-valued encoding). Κάθε άτομο (χρωμόσωμα) αναπαρίσταται ως ένα διάνυσμα πραγματικών αριθμών που περιέχει τις παραμέτρους των $K=15$ Γκαουσιανών συναρτήσεων.

Συγκεκριμένα, για κάθε Γκαουσιανή $k$, απαιτούνται 5 παράμετροι: το βάρος $w_k$, τα κέντρα $c_{1,k}, c_{2,k}$ και οι τυπικές αποκλίσεις $\sigma_{1,k}, \sigma_{2,k}$. Επομένως, το μήκος του χρωμοσώματος είναι $L = 5 \times 15 = 75$ γονίδια. Η δομή του χρωμοσώματος έχει τη μορφή:
\begin{equation}
    \mathbf{x} = [w_1, c_{1,1}, \sigma_{1,1}, c_{2,1}, \sigma_{2,1}, \dots, w_{15}, c_{1,15}, \sigma_{1,15}, c_{2,15}, \sigma_{2,15}]
\end{equation}

\subsection{Συνάρτηση Αξιολόγησης}
Η αξιολόγηση της ποιότητας κάθε λύσης γίνεται μέσω του Μέσου Τετραγωνικού Σφάλματος (MSE) μεταξύ της πραγματικής συνάρτησης $f(u_1, u_2)$ και της εκτίμησης $G(u_1, u_2)$ στα δεδομένα εκπαίδευσης.

Επειδή οι γενετικοί αλγόριθμοι συνήθως μεγιστοποιούν μια συνάρτηση ικανότητας (Fitness Function), ενώ εμείς επιθυμούμε την ελαχιστοποίηση του σφάλματος, χρησιμοποιήθηκε ο ακόλουθος μετασχηματισμός στη συνάρτηση \texttt{fitness}:
\begin{equation}
    Fitness(\mathbf{x}) = \frac{1}{1 + MSE(\mathbf{x})}
\end{equation}
Με αυτόν τον τρόπο, όσο μικρότερο είναι το σφάλμα (MSE $\to$ 0), τόσο η τιμή Fitness πλησιάζει τη μονάδα (μέγιστη τιμή).

\subsection{Γενετικοί Τελεστές}
\begin{itemize}
    \item \textbf{Επιλογή (Selection):} Χρησιμοποιήθηκε η μέθοδος του Τροχού της Τύχης (Roulette Wheel Selection). Σύμφωνα με τη μέθοδο αυτή, κάθε άτομο έχει πιθανότητα επιλογής ανάλογη της ικανότητάς του. Η πιθανότητα $P_i$ για το άτομο $i$ υπολογίζεται ως εξής:
    \begin{equation}
        P_i = \frac{f_i}{\sum_{j=1}^{N} f_j}
    \end{equation}
    όπου $f_i$ η τιμή της συνάρτησης ικανότητας του ατόμου.
    
    Ωστόσο, καθώς προχωρούν οι γενεές και ο γενετικός αλγόριθμος πλησιάζει στη σύγκλιση, η ικανότητα των ατόμων τείνει να έχει την ίδια περίπου τιμή. Κατά συνέπεια, όλα τα άτομα αποκτούν σχεδόν την ίδια πιθανότητα επιβίωσης και ο γενετικός αλγόριθμος κινδυνεύει να εκφυλιστεί σε μέθοδο τυχαίας αναζήτησης.
    
    Για την επίλυση αυτού του προβλήματος, εφαρμόστηκε \textbf{κλιμάκωση (scaling)} της συνάρτησης ικανότητας, ώστε η υψηλότερη τιμή του τρέχοντος πληθυσμού να απεικονίζεται στο 1 και η χαμηλότερη στο 0 (όπως υλοποιείται στη συνάρτηση \texttt{selection}).
    
    \item \textbf{Διασταύρωση (Crossover):} Εφαρμόστηκε αριθμητική διασταύρωση με πιθανότητα 0.8. Δύο γονείς $p_1, p_2$ παράγουν απογόνους μέσω γραμμικού συνδυασμού:
    $c_1 = \alpha \cdot p_1 + (1-\alpha) \cdot p_2$ και $c_2 = (1-\alpha) \cdot p_1 + \alpha \cdot p_2$, όπου $\alpha$ τυχαίος αριθμός στο $(0,1)$.
    
    \item \textbf{Μετάλλαξη (Mutation):} Εφαρμόστηκε μετάλλαξη με προσθήκη Γκαουσιανού θορύβου (Gaussian Mutation) με πιθανότητα 0.05 ανά γονίδιο. Στις νέες τιμές επιβάλλονται ξανά τα όρια (clamping) ώστε να παραμένουν εντός των επιτρεπτών ορίων του προβλήματος.
\end{itemize}

\newpage
\section{Αποτελέσματα}
Τα παρακάτω αποτελέσματα προέκυψαν τρέχοντας τον αλγόριθμο για \texttt{max\_generation} γενιές.
\subsection{Γραφικές Παραστάσεις}

    \begin{figure}[H]
        \centering
        \includegraphics[width=0.9\linewidth]{assets/main_3.jpg}
        \caption{Σύγκλιση του αλγορίθμου ανά τις γενιές}
        \label{fig:convergence}
    \end{figure}

    \begin{figure}[H]
        \centering
        \includegraphics[width=0.9\linewidth]{assets/main_4.jpg}
        \caption{Οπτική σύγκριση της πραγματικής συνάρτησης με την εκτίμηση}
        \label{fig:comparison}
    \end{figure}

\subsection{Ευρήματα}
Ο Γενετικός Αλγόριθμος κατέληξε σε ένα σύνολο 15 Γκαουσιανών συναρτήσεων. Στον Πίνακα \ref{tab:final_params} παρουσιάζονται αναλυτικά οι βέλτιστες τιμές των παραμέτρων ($w_k, c_{1,k}, \sigma_{1,k}, c_{2,k}, \sigma_{2,k}$) που συνθέτουν την τελική αναλυτική έκφραση.

\begin{table}[H]
    \centering
    \small
    \begin{tabular}{|c|r|rr|rr|}
    \hline
    \textbf{k} & \textbf{$w_k$} & \textbf{$c_{1,k}$} & \textbf{$\sigma_{1,k}$} & \textbf{$c_{2,k}$} & \textbf{$\sigma_{2,k}$} \\ \hline
    1 & 2.225 & 0.56 & 2.00 & -0.68 & 2.16 \\
    2 & -1.709 & 0.82 & 1.62 & -0.88 & 1.99 \\
    3 & 2.196 & 0.85 & 1.83 & -0.27 & 1.81 \\
    4 & -0.763 & 0.78 & 2.19 & -0.52 & 1.61 \\
    5 & -1.558 & 0.42 & 1.61 & -0.10 & 1.14 \\
    ... & ... & ... & ... & ... & ... \\
    15 & -0.720 & -0.36 & 2.39 & -0.32 & 1.82 \\ \hline
    \end{tabular}
    \caption{Οι βέλτιστες παράμετροι των 15 Γκαουσιανών όρων της τελικής λύσης.}
    \label{tab:final_params}
\end{table}

Η τελική προσεγγιστική συνάρτηση $\hat{f}(u_1, u_2)$ δίνεται από το άθροισμα των όρων του Πίνακα \ref{tab:final_params}. Ενδεικτικά, ο όρος με τη μεγαλύτερη επιρροή (μεγαλύτερο πλάτος $|w| = 2.225$) είναι:
\begin{equation}
    G_1(u_1, u_2) = 2.225 \cdot e^{-\left(\frac{(u_1-0.56)^2}{2(2.00)^2} + \frac{(u_2+0.68)^2}{2(2.16)^2}\right)}
\end{equation}
Η πλήρης αναλυτική έκφραση προκύπτει αθροίζοντας και τους 15 όρους.

\subsection{Παρατηρήσεις}
Από την εκτέλεση του αλγορίθμου και την ανάλυση των αποτελεσμάτων, προκύπτουν τα εξής συμπεράσματα:

\begin{enumerate}
    \item \textbf{Σύγκλιση:} Όπως φαίνεται στο Σχήμα \ref{fig:convergence}, ο αλγόριθμος επιτυγχάνει ταχεία μείωση του σφάλματος στις πρώτες 500 γενιές, ενώ στη συνέχεια πραγματοποιεί μικρο-βελτιώσεις (fine-tuning) μέχρι τη γενιά 3000. Το τελικό MSE εκπαίδευσης ήταν \textit{0.00733}.
    
    \item \textbf{Ικανότητα Γενίκευσης:} Κατά τη φάση της επαλήθευσης (validation) με νέα δεδομένα, το σφάλμα ήταν \textbf{0.00753}, τιμή πολύ κοντινή στο σφάλμα εκπαίδευσης. Αυτό υποδεικνύει ότι το μοντέλο δεν έχει υποστεί υπερπροσαρμογή (overfitting) και γενικεύει σωστά σε άγνωστα δεδομένα.
    
    \item \textbf{Ποιότητα Προσέγγισης:} Η οπτική σύγκριση στο Σχήμα \ref{fig:comparison} επιβεβαιώνει ότι η προσεγγιστική συνάρτηση ακολουθεί πιστά την καμπυλότητα της πραγματικής συνάρτησης $f(u_1, u_2)$, ακόμα και στις περιοχές με έντονες διακυμάνσεις. Η ανακατασκευή είναι εν πολλοίς επιτυχής, αφού φαίνεται να προσεγγίζει πιστά την αρχική συνάρτηση, παρά την ιδιαίτερη μορφή της.
    
    \item \textbf{Απλοποίηση Μοντέλου:} Εφαρμόστηκε διαδικασία απλοποίησης (pruning) με κατώφλι βάρους $|w| < 0.1$. Στην προκειμένη περίπτωση, και οι 15 Γκαουσιανοί όροι διατηρήθηκαν (Active Gaussians: 15), καθώς όλοι είχαν σημαντική συνεισφορά (βάρη $>0.1$) στη διαμόρφωση της τελικής λύσης.

    \item \textbf{Χρόνος Εκτέλεσης:} Ο αλγόριθμος φαίνεται να είναι αποδοτικός, όσον αφορά τον χρόνο. Παρά το μεγάλο πλήθος των παραμέτρων που απαιτούν ρύθμιση, ο αλγόριθμος τερμάτισε σε \textit{34.41 δευτερόλεπτα}.
\end{enumerate}

\newpage
\appendix

\section{Εργαλεία}
Τα εργαλεία που χρησιμοποιήθηκαν κατά την εκπόνηση της εργασίας είναι τα εξής:
\begin{itemize}
    \item MATLAB
    \item LaTeX
    \item Git
\end{itemize}

\begin{thebibliography}{9}

    \bibitem[\href{https://www.tziola.gr/book/rovi/}{1}]{book}
    Γ. Ροβιθάκης, \textit{Τεχνικές Βελτιστοποίησης}, Εκδόσεις Τζιόλα, 2007.
    
    \bibitem[\href{https://www2.fiit.stuba.sk/~kvasnicka/Free\%20books/Goldberg_Genetic_Algorithms_in_Search.pdf}{2}]{paper}
    D. E. Goldberg, \textit{Genetic Algorithms in Search, Optimization, and Machine Learning}. Reading, MA: Addison-Wesley, 1989.
    
    \bibitem[\href{https://www.mathworks.com/help/gads/what-is-the-genetic-algorithm.html}{3}]{mathworks}
    MathWorks, What Is the Genetic Algorithm?, \textit{MathWorks Documentation}.
    
    \bibitem[\href{https://medium.com/@sanskar4862/how-to-generate-a-genetic-algorithm-with-matlab-6dbc595d54c8}{4}]{medium}
    S. Sanskar, How to Generate a Genetic Algorithm with MATLAB, \textit{Medium}, Dec. 2024.
    
    \bibitem[\href{https://www.generativedesign.org/02-deeper-dive/02-04_genetic-algorithms}{5}]{gendesign}
    Generative Design, Genetic Algorithms, \textit{Generative Design Library}.

\end{thebibliography}

\end{document}