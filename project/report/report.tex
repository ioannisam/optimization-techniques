\documentclass[a4paper,12pt]{article}

\usepackage{fontspec}
\usepackage{microtype}
\usepackage{polyglossia}

\setmainlanguage{greek}
\setotherlanguage{english}

\setmainfont{Latin Modern Roman}
\newfontfamily\greekfont{CMU Serif}[Script=Greek]

\usepackage[a4paper, top=1.5cm, bottom=1.5cm, left=1.5cm, right=1.5cm]{geometry}
\usepackage{graphicx}
\usepackage{float}
\usepackage{array}
\usepackage{hyperref}
\hypersetup{
    colorlinks=true,
    linkcolor=black,
    citecolor=black,
    urlcolor=blue
}
\usepackage[super,sort&compress]{cite}
\usepackage{amsmath}
\usepackage{algorithm}
\usepackage{algpseudocode}

\title{Τεχνικές Βελτιστοποίησης: Project}
\author{Ιωάννης Μιχάλαινας ΑΕΜ:10902}
\date{Δεκέμβριος 2025}

\begin{document}

\maketitle

\begin{center}
\textbf{Αποθετήριο Κώδικα:} \\
\href{https://github.com/ioannisam/optimization-techniques}
     {github.com/ioannisam/optimization-techniques}
\end{center}

\bigskip

\begin{abstract}
    Η παρούσα εργασία πραγματεύεται την προσέγγιση μιας άγνωστης μη γραμμικής συνάρτησης δύο μεταβλητών, $f(u_1, u_2)$, μέσω ενός Γενετικού Αλγορίθμου. Στόχος είναι η εύρεση μιας αναλυτικής έκφρασης που αποτελείται από γραμμικό συνδυασμό Γκαουσιανών συναρτήσεων. Ο αλγόριθμος υλοποιήθηκε εξ ολοκλήρου στο MATLAB, χωρίς τη χρήση έτοιμων εργαλείων βελτιστοποίησης. Η διαδικασία περιλαμβάνει την εκπαίδευση του μοντέλου, την επαλήθευση σε νέα δεδομένα για την αποφυγή υπερπροσαρμογής και τέλος, μια διαδικασία απλοποίησης για τη μείωση της πολυπλοκότητας της τελικής έκφρασης.
\end{abstract}

\tableofcontents
\clearpage

\section{Εισαγωγή}
Στο πλαίσιο της εργασίας, ζητείται η εκτίμηση της συνάρτησης $y=f(u_1, u_2)$ βάσει μετρήσεων εισόδου-εξόδου. Ως μοντέλο προσέγγισης χρησιμοποιείται το άθροισμα $K=15$ Γκαουσιανών όρων της μορφής:
\begin{equation}
G(u_1, u_2) = w \cdot e^{-\left(\frac{(u_1-c_1)^2}{2\sigma_1^2} + \frac{(u_2-c_2)^2}{2\sigma_2^2}\right)}
\end{equation}
Η πραγματική συνάρτηση που προσπαθούμε να προσεγγίσουμε και που χρησιμοποιήθηκε για την παραγωγή των δεδομένων εκπαίδευσης και επαλήθευσης είναι η: $$f(u_1,u_2)=\sin(u_1+u_2)\sin(u_2^2)$$

\begin{figure}[H]
    \centering
    \begin{minipage}{0.48\linewidth}
        \centering
        \includegraphics[width=\linewidth]{assets/main_1.jpg}
        \caption{Οπτικοποίηση της $f(u_1,u_2)$ \\ (100 δείγματα)}
        \label{fig:visualization1}
    \end{minipage}
    \hfill
    \begin{minipage}{0.48\linewidth}
        \centering
        \includegraphics[width=\linewidth]{assets/main_2.jpg}
        \caption{Οπτικοποίηση της $f(u_1,u_2)$ \\ (5000 δείγματα)}
        \label{fig:visualization2}
    \end{minipage}
\end{figure}


\newpage
\section{Γενετικοί Αλγόριθμοι}
\subsection{Θεωρία}
\subsection{Υλοποίηση}
    \subsubsection{Κωδικοποίηση και Αρχικοποίηση}
    \subsubsection{Συνάρτηση Αξιολόγησης}
    \subsubsection{Γενετικοί Τελεστές}
    \begin{itemize}
        \item \textbf{Επιλογή (Selection):}
        \item \textbf{Διασταύρωση (Crossover):}
        \item \textbf{Μετάλλαξη (Mutation):}
    \end{itemize}

\newpage
\section{Αποτελέσματα}
\subsection{Γραφικές Παραστάσεις}

    \begin{figure}[H]
        \centering
        \includegraphics[width=0.9\linewidth]{assets/main_3.jpg}
        \caption{Σύγκλιση του αλγορίθμου ανά τις γενιές}
        \label{fig:convergence}
    \end{figure}

    \begin{figure}[H]
        \centering
        \includegraphics[width=0.9\linewidth]{assets/main_4.jpg}
        \caption{Οπτική σύγκριση της πραγματικής συνάρτησης με την εκτίμηση}
        \label{fig:comparison}
    \end{figure}
\subsection{Παρατηρήσεις}

\newpage
\appendix

\section{Εργαλεία}
Τα εργαλεία που χρησιμοποιήθηκαν κατά την εκπόνηση της εργασίας είναι τα εξής:
\begin{itemize}
    \item MATLAB
    \item LaTeX
    \item Git
\end{itemize}

\begin{thebibliography}{9}

    \bibitem[\href{https://www.tziola.gr/book/rovi/}{1}]{book}
    Γ. Ροβιθάκης, \textit{Τεχνικές Βελτιστοποίησης}, Εκδόσεις Τζιόλα, 2007.
    
    \bibitem[\href{https://www2.fiit.stuba.sk/~kvasnicka/Free\%20books/Goldberg_Genetic_Algorithms_in_Search.pdf}{2}]{paper}
    D. E. Goldberg, \textit{Genetic Algorithms in Search, Optimization, and Machine Learning}. Reading, MA: Addison-Wesley, 1989.
    
    \bibitem[\href{https://www.mathworks.com/help/gads/what-is-the-genetic-algorithm.html}{3}]{mathworks}
    MathWorks, What Is the Genetic Algorithm?, \textit{MathWorks Documentation}.
    
    \bibitem[\href{https://medium.com/@sanskar4862/how-to-generate-a-genetic-algorithm-with-matlab-6dbc595d54c8}{4}]{medium}
    S. Sanskar, How to Generate a Genetic Algorithm with MATLAB, \textit{Medium}, Dec. 2024.
    
    \bibitem[\href{https://www.generativedesign.org/02-deeper-dive/02-04_genetic-algorithms}{5}]{gendesign}
    Generative Design, Genetic Algorithms, \textit{Generative Design Library}.

\end{thebibliography}

\end{document}