\documentclass[a4paper,12pt]{article}

\usepackage{fontspec}
\usepackage{microtype}
\usepackage{polyglossia}

\setmainlanguage{greek}
\setotherlanguage{english}

\setmainfont{Latin Modern Roman}
\newfontfamily\greekfont{CMU Serif}[Script=Greek]

\usepackage[a4paper, top=1.5cm, bottom=1.5cm, left=1.5cm, right=1.5cm]{geometry}
\usepackage{graphicx}
\usepackage{float}
\usepackage{array}
\usepackage{hyperref}
\hypersetup{
    colorlinks=true,
    linkcolor=black,
    citecolor=black,
    urlcolor=blue
}
\usepackage[super,sort&compress]{cite}
\usepackage{amsmath}
\usepackage{algorithm}
\usepackage{algpseudocode}

\usepackage{tabularx}
\usepackage{makecell}
\newcolumntype{Y}{>{\centering\arraybackslash}X}

\newcolumntype{C}{>{\centering\arraybackslash}c}

\title{Τεχνικές Βελτιστοποίησης: Εργασία 1}
\author{Ιωάννης Μιχάλαινας ΑΕΜ:10902}
\date{Νοέμβριος 2025}

\begin{document}

\maketitle

\begin{center}
\textbf{Αποθετήριο Κώδικα:} \\
\href{https://github.com/ioannisam/optimization-techniques}
     {github.com/ioannisam/optimization-techniques}
\end{center}

\bigskip

\begin{abstract}
    Στην εργασία αυτή υλοποιούνται και συγκρίνονται μονοδιάστατες μέθοδοι βελτιστοποίησης - συγκεκριμένα η μέθοδος Διχοτόμου, η μέθοδος Χρυσού Τομέα, η μέθοδος Fibonacci και η Διχοτόμος Μέθοδος με χρήση παραγώγου. Για καθεμία από αυτές εξετάζεται ο αριθμός υπολογισμών και η ταχύτητα σύγκλισης, εφαρμόζοντάς τες σε διαφορετικές συναρτήσεις δοκιμής. Τα αποτελέσματα δείχνουν τις ομοιότητες και τις διαφορές μεταξύ των μεθόδων και αναδεικνύουν τα πλεονεκτήματα και τους περιορισμούς της καθεμιάς. Ο σκοπός της εργασίας αυτής είναι η εξοικείωση με τις παραπάνω μεθόδους και η σύγκριση μεταξύ τους, όσον αφορά την αποδοτικότητα.
\end{abstract}

\tableofcontents
\clearpage

\section{Εισαγωγή}
Στην εργασία αυτή μελετάμε διάφορες τεχνικές βελτιστοποίησης, ονομαστικά τη μέθοδο Διχοτόμου, τη μέθοδο Χρυσού Τομέα, τη μέθοδο Fibonacci και τη μέθοδο Διχοτόμου με χρήση παραγώγου. Συγκεκριμένα, θα ασχοληθούμε μόνο με προβλήματα ελαχιστοποίησης σχεδόν κυρτών συναρτήσεων, αφού η διαδικασία για προβλήματα μεγιστοποίησης είναι αντίστοιχη (κοίλες συναρτήσεις). Ο σκοπός είναι ο περιορισμός του διαστήματος στο οποίο ανήκει το $x^*$, η λύση του προβλήματος ελαχιστοποίησης.

Οι μέθοδοι που θα εξετάσουμε χωρίζονται σε δύο κατηγορίες: σε αυτές χωρίς και σε αυτές με τη χρήση παραγώγων. Οι τεχνικές της κατηγορίας της πρώτης χωρίζονται σε δύο κλάσεις, την ταυτόχρονη, όπου τα υποψήφια σημεία ελαχίστου προσδιορίζονται εκ των προτέρων, και την ακολουθιακή, όπου οι τιμές της συνάρτησης που προέκυψαν σε προηγούμενες επαναλήψεις χρησιμοποιούνται για τον προσδιορισμό των επόμενων σημείων.

Δοσμένου του αρχικού διαστήματος αναζήτησης $[-1,3]$, θα ελαχιστοποιηθούν οι:
\begin{itemize}
    \item $f_1(x) = 5^x + (2-\cos(x))^2$
    \item $f_2(x) = (x-1)^2 + e^{x-5} \cdot \sin(x+3)$
    \item $f_3(x) = x^{-3x} - \big(\sin(x-2) - 2\big)^2$
\end{itemize}

Με βάση τη γραφική απεικόνιση, οι συναρτήσεις φαίνεται να συμπεριφέρονται ως σχεδόν κυρτές στο διάστημα [-1,3].

\begin{figure}[H]
    \centering
    \includegraphics[width=1\linewidth]{assets/visualization.jpg}
    \caption{Οπτικοποίηση των αντικειμενικών συναρτήσεων}
    \label{fig:visualization}
\end{figure}

\newpage
\section{Μέθοδος Διχοτόμου}

    \subsection{Αλγόριθμος}
    
    Η μέθοδος διχοτόμου ανήκει στην ακολουθιακή κλάση και χωρίζει επαναληπτικά το διάστημα αναζήτησης σε τρία (ανισομερή) υποδιαστήματα χρησιμοποιώντας δύο εσωτερικά σημεία $x_1$ και $x_2$, που βρίσκονται συμμετρικά γύρω από το μέσο του διαστήματος. Σύμφωνα με τη σύγκριση των τιμών $f(x_1)$ και $f(x_2)$, το διάστημα αναζήτησης περιορίζεται στο αριστερό ή δεξί υποδιάστημα, με βάση το Θεώρημα 5.1.1 του βιβλίου \cite{book}. Η διαδικασία επαναλαμβάνεται έως ότου το διάστημα γίνει μικρότερο από μια προκαθορισμένη ακρίβεια $l$.
    
    \begin{algorithm}[H]
        \caption{Μέθοδος Διχοτόμου}
        \begin{algorithmic}[1]
        \Require συνάρτηση $f$, διάστημα $[a_1,b_1]$, σταθερά $\epsilon>0$, τελική ακρίβεια $l>0$
        \Ensure προσέγγιση ελαχίστου $x^*$
        \State ορίστε $a \gets a_1$, $b \gets b_1$, $k \gets 1$
        \While{$b - a > l$}
            \State $m \gets (a+b)/2$
            \State $x_1 \gets m - \epsilon$
            \State $x_2 \gets m + \epsilon$
            \State $f_1 \gets f(x_1)$
            \State $f_2 \gets f(x_2)$
            \If{$f_1 < f_2$}
                \State $b \gets x_2$ \Comment{Το ελάχιστο βρίσκεται στο $[a, x_2]$}
            \Else
                \State $a \gets x_1$ \Comment{Το ελάχιστο βρίσκεται στο $[x_1, b]$}
            \EndIf
            \State $k \gets k+1$
        \EndWhile
        
        \State \Return $[a_K,b_K] \gets [a,b]$
        \end{algorithmic}
    \end{algorithm}
    
    \subsection{Γραφικές Παραστάσεις}

    Παρακάτω ακολουθούν οι γραφικές παραστάσεις της ανάλυσης που διεξαγάγαμε. 
    \begin{itemize}
        \item Στο \textit{Σχήμα~\ref{fig:task1_1}} παρουσιάζεται η μεταβολή των υπολογισμών της αντικειμενικής συνάρτησης $f_i$, για δεδομένη τελική ακρίβεια $l$ και μεταβαλλόμενη σταθερά $\epsilon$.
        \item Στο \textit{Σχήμα~\ref{fig:task1_2}} παρουσιάζεται η μεταβολή των υπολογισμών της αντικειμενικής συνάρτησης $f_i$, για δεδομένη σταθερά $\epsilon$ και μεταβαλλόμενη τελική ακρίβεια $l$.
        \item Στο \textit{Σχήμα~\ref{fig:task1_3}} παρουσιάζονται τα άκρα του διαστήματος $[a_k,b_k]$ συναρτήσει του δείκτη επαναλήψεων $k$ για διάφορα $l$.
    \end{itemize}  
    
    \begin{figure}[H]
        \centering
        \includegraphics[width=0.75\linewidth]{assets/task1_1.jpg}
        \caption{Υπολογισμοί συνάρτησης για διαφορετικά $\epsilon$}
        \label{fig:task1_1}
    \end{figure}

    \begin{figure}[H]
        \centering
        \includegraphics[width=0.75\linewidth]{assets/task1_2.jpg}
        \caption{Υπολογισμοί συνάρτησης για διαφορετικά $l$}
        \label{fig:task1_2}
    \end{figure}

    \begin{figure}[H]
        \centering
        \includegraphics[width=0.75\linewidth]{assets/task1_3.jpg}
        \caption{Άκρα του διαστήματος $[a_k,b_k]$ για διαφορετικά $k$ και $l$}
        \label{fig:task1_3}
    \end{figure}
    
    \subsection{Συμπεράσματα}

    \textbf{Ερώτημα 1:} Παρατηρούμε πως όσο αυξάνεται η σταθερά $\epsilon$ τόσο αυξάνονται οι υπολογισμοί της αντικειμενικής συνάρτησης. Αυτό εξηγείται θεωρητικά, καθώς η σταθερά $\epsilon$ ορίζει την απόσταση από τη διχοτόμο του εκάστοτε διαστήματος. Συνεπώς, η αύξησή του συνεπάγεται περισσότερες επαναλήψεις του αλγορίθμου και κατ' επέκταση περισσότερους υπολογισμούς αντικειμενικής συνάρτησης. Μάλιστα, παρατηρούμε πως ο αριθμός υπολογισμών της αντικειμενικής συνάρτησης δεν εξαρτάται από την ίδια την συνάρτηση, αλλά μόνο από το $l$, το $\epsilon$ και το αρχικό διάστημα $[a,b]$, αφού οι καμπύλες συμπίπτουν, πράγμα που συνάδει με τον θεωρητικό τύπο των υπολογισμών συναρτήσεων\cite{book}, ήτοι: $$(1/2)^n \le {l \over b_1-a_1}$$ 
    
    \textbf{Ερώτημα 2:} Παρατηρούμε πως όσο αυξάνεται το εύρος $l$ τόσο μειώνονται οι υπολογισμοί της αντικειμενικής συνάρτησης. Αυτό εξηγείται θεωρητικά, καθώς το εύρος $l$ συσχετίζεται με την ακρίβεια που επιζητούμε από τον αλγόριθμο. Συνεπώς, η αύξησή του καθιστά τον αλγόριθμο πιο ελαφρύ, αφού λιγότερες επαναλήψεις απαιτούνται, και το επιστρεφόμενο διάστημα $[a_k,b_k]$ που περιλαμβάνει το $x^*$ είναι πιο ευρύ. Πάλι δεν παρατηρούμε διαφορά στα γραφήματα στις διαφορετικές συναρτήσεις, όπως στο Ερώτημα 1.
    
    \textbf{Ερώτημα 3:} Παρατηρούμε πως η μεταβολή του $l$ δεν επηρεάζει τη μορφή της γραφικής παράστασης του διαστήματος $[a_k,b_k]$ για κάποια δεδομένη συνάρτηση $f_i$. Η μόνη διαφορά που προκαλεί είναι πως αυξάνει τον αριθμό επαναλήψεων του αλγορίθμου και συνεπώς τη λεπτομέρεια του αλγορίθμου, εύρημα που συμφωνεί με αυτά του Ερωτήματος 2. Για διαφορετικές συναρτήσεις παρατηρούμε διαφορετική γραφική παράσταση του διαστήματος, το οποίο είναι λογικό, αφού διαφέρουν μεταξύ τους ως προς τη μορφή και την κυρτότητα. Τα διαστήματα $[a_k,b_k]$ που προκύπτουν και περιλαμβάνουν το $x^*$ συμφωνούν με την οπτική απεικόνιση των δεδομένων στο \textit{Σχήμα~\ref{fig:visualization}}.

\newpage
\section{Μέθοδος Χρυσού Τομέα}

    \subsection{Αλγόριθμος}

    Στη μέθοδο του χρυσού τομέα, τα εσωτερικά σημεία $x_1$ και $x_2$ επιλέγονται σύμφωνα με τον χρυσό λόγο $\gamma \approx 0.618$, ώστε σε κάθε επανάληψη το διάστημα αναζήτησης να μειώνεται κατά έναν σταθερό παράγοντα. Ένα από τα σημεία επαναχρησιμοποιείται στην επόμενη επανάληψη, μειώνοντας τον αριθμό των απαραίτητων υπολογισμών της συνάρτησης. Η μέθοδος συγκλίνει γρήγορα και απαιτεί μόνο έναν νέο υπολογισμό ανά επανάληψη μετά την πρώτη.
    
    \begin{algorithm}[H]
        \caption{Μέθοδος Χρυσού Τομέα}
        \begin{algorithmic}[1]
        \Require συνάρτηση $f$, διάστημα $[a_1,b_1]$, τελική ακρίβεια $l>0$
        \Ensure προσέγγιση ελαχίστου $x^*$
        \State ορίστε $a \gets a_1$, $b \gets b_1$, $k \gets 1$
        \State $\gamma \gets (\sqrt{5}-1)/2$ \Comment{Χρυσός Λόγος (0.618)}
        \State $x_1 \gets a + (1-\gamma)*(b-a)$
        \State $x_2 \gets a + \gamma*(b-a)$
        \State $f_1 \gets f(x_1)$
        \State $f_2 \gets f(x_2)$
        \While{$b - a > l$}
            \If{$f_1 > f_2$}
                \State $a \gets x_1$
                \State $x_1 \gets x_2$
                \State $f_1 \gets f_2$
                \State $x_2 \gets a + \gamma*(b-a)$
                \State $f_2 \gets f(x_2)$
            \Else
                \State $b \gets x_2$
                \State $x_2 \gets x_1$
                \State $f_2 \gets f_1$
                \State $x_1 \gets a + (1-\gamma)*(b-a)$
                \State $f_1 \gets f(x_1)$ 
            \EndIf
            \State $k \gets k + 1$
        \EndWhile
        
        \State \Return $[a_K,b_K] \gets [a,b]$
        \end{algorithmic}
    \end{algorithm}

    \subsection{Γραφικές Παραστάσεις}

    Παρακάτω ακολουθούν οι γραφικές παραστάσεις της ανάλυσης που διεξαγάγαμε. 
    \begin{itemize}
        \item Στο \textit{Σχήμα~\ref{fig:task2_1}} παρουσιάζεται η μεταβολή των υπολογισμών της αντικειμενικής συνάρτησης $f_i$, για μεταβαλλόμενη τελική ακρίβεια $l$.
        \item Στο \textit{Σχήμα~\ref{fig:task2_2}} παρουσιάζονται τα άκρα του διαστήματος $[a_k,b_k]$ συναρτήσει του δείκτη επαναλήψεων $k$ για διάφορα $l$.
    \end{itemize}  

    \begin{figure}[H]
        \centering
        \includegraphics[width=0.75\linewidth]{assets/task2_1.jpg}
        \caption{Υπολογισμοί συνάρτησης για διαφορετικά $l$}
        \label{fig:task2_1}
    \end{figure}

    \begin{figure}[H]
        \centering
        \includegraphics[width=0.75\linewidth]{assets/task2_2.jpg}
        \caption{Άκρα του διαστήματος $[a_k,b_k]$ για διαφορετικά $k$ και $l$}
        \label{fig:task2_2}
    \end{figure}
    
    \subsection{Συμπεράσματα}

    \textbf{Ερώτημα 1:} Τα ευρήματα είναι παρόμοια με αυτά της Μεθόδου της Διχοτόμου. Ο θεωρητικός τύπος\cite{book} για τις επαναλήψεις είναι: $$(\gamma)^n \le {l \over b_1-a_1}$$ όπου $\gamma \approx 0.618$ η χρυσή τομή.
    
    \textbf{Ερώτημα 2:} Τα ευρήματα είναι παρόμοια με αυτά της Μεθόδου της Διχοτόμου.

\newpage
\section{Μέθοδος Fibonacci}

    \subsection{Αλγόριθμος}

    Η μέθοδος Fibonacci χρησιμοποιεί την ακολουθία Fibonacci για τον προσδιορισμό των σημείων $x_1$ και $x_2$. Ανήκει στην ταυτόχρονη κλάση, και συνεπώς, ο αριθμός των επαναλήψεων πρέπει να καθοριστεί εκ των προτέρων. Το διάστημα αναζήτησης μειώνεται σε κάθε βήμα με διαφορετικό λόγο, ο οποίος εξαρτάται από τους αριθμούς Fibonacci. Η μέθοδος είναι βέλτιστη ως προς τον αριθμό των υπολογισμών για δεδομένη ακρίβεια.

    \begin{algorithm}[H]
        \caption{Μέθοδος Fibonacci}
        \begin{algorithmic}[1]
        \Require συνάρτηση $f$, διάστημα $[a_1,b_1]$, σταθερά $\epsilon>0$, τελική ακρίβεια $l>0$
        \Ensure προσέγγιση ελαχίστου $x^*$
        \State Ορίστε $a \gets a_1$, $b \gets b_1$, $k \gets 1$
        \State $F = [1, 1]$ \Comment{Σειρά Fibonacci}
        \While{$F_{\text{end}} < (b-a)/l$}
            \State Προσθέστε $F_{\text{end}} + F_{\text{end-1}}$ στο τέλος της σειράς
        \EndWhile
        \State $n \gets \text{μήκος}(F)$
        \State $x_1 \gets a + (F_{n-2}/F_n)*(b-a)$
        \State $x_2 \gets a + (F_{n-1}/F_n)*(b-a)$
        \State $f_1 \gets f(x_1)$
        \State $f_2 \gets f(x_2)$
        
        \For{$k = 1$ to $n-3$}
            \If{$f_1 > f_2$}
                \State $a \gets x_1$
                \State $x_1 \gets x_2$
                \State $f_1 \gets f_2$
                \State $x_2 \gets a + (F_{n-k-1}/F_{n-k})*(b-a)$
                \State $f_2 \gets f(x_2)$
            \Else
                \State $b \gets x_2$
                \State $x_2 \gets x_1$
                \State $f_2 \gets f_1$
                \State $x_1 \gets a + (F_{n-k-2}/F_{n-k})*(b-a)$
                \State $f_1 \gets f(x_1)$
            \EndIf
        \EndFor

        \State $x_2 \gets x_1 + \epsilon$ 
        \If{$f_1 > f_2$}
            \State $a \gets x_1$
        \Else
            \State $b \gets x_2$
        \EndIf
        
        \State \Return $[a_K,b_K] \gets [a,b]$
        \end{algorithmic}
    \end{algorithm}

    \subsection{Γραφικές Παραστάσεις}

    Παρακάτω ακολουθούν οι γραφικές παραστάσεις της ανάλυσης που διεξαγάγαμε. 
    \begin{itemize}
        \item Στο \textit{Σχήμα~\ref{fig:task3_1}} παρουσιάζεται η μεταβολή των υπολογισμών της αντικειμενικής συνάρτησης $f_i$, για μεταβαλλόμενη τελική ακρίβεια $l$.
        \item Στο \textit{Σχήμα~\ref{fig:task3_2}} παρουσιάζονται τα άκρα του διαστήματος $[a_k,b_k]$ συναρτήσει του δείκτη επαναλήψεων $k$ για διάφορα $l$.
    \end{itemize}

    \begin{figure}[H]
        \centering
        \includegraphics[width=0.75\linewidth]{assets/task3_1.jpg}
        \caption{Υπολογισμοί συνάρτησης για διαφορετικά $l$}
        \label{fig:task3_1}
    \end{figure}

    \begin{figure}[H]
        \centering
        \includegraphics[width=0.75\linewidth]{assets/task3_2.jpg}
        \caption{Άκρα του διαστήματος $[a_k,b_k]$ για διαφορετικά $k$ και $l$}
        \label{fig:task3_2}
    \end{figure}
    
    \subsection{Συμπεράσματα}

    \textbf{Ερώτημα 1:} Τα ευρήματα είναι παρόμοια με αυτά της Μεθόδου της Διχοτόμου. Ο θεωρητικός τύπος\cite{book} για τις επαναλήψεις είναι: $$F_n \ge {b_1-a_1 \over l}$$
    
    \textbf{Ερώτημα 2:} Τα ευρήματα είναι παρόμοια με αυτά της Μεθόδου της Διχοτόμου.

\section{Μέθοδος Διχοτόμου (με χρήση παραγώγου)}

    \subsection{Αλγόριθμος}

    Σε αυτήν την παραλλαγή της μεθόδου διχοτόμου, χρησιμοποιείται η παράγωγος $f'(x)$ στο μέσο του διαστήματος. Αν $f'(x) > 0$, το ελάχιστο βρίσκεται στο αριστερό μισό, ενώ αν $f'(x) < 0$, βρίσκεται στο δεξί μισό. Η μέθοδος συγκλίνει εκθετικά και απαιτεί τον υπολογισμό της παραγώγου σε κάθε βήμα.

    \begin{algorithm}[H]
    \caption{Μέθοδος Διχοτόμου (με χρήση παραγώγου)}
    \begin{algorithmic}[1]
        \Require παράγωγος συνάρτησης $f'$, διάστημα $[a_1,b_1]$, τελική ακρίβεια $l>0$
        \Ensure τελικό διάστημα $[a_K,b_K]$ που περιέχει το ελάχιστο
    
        \State Ορίστε $a \gets a_1$, $b \gets b_1$
        \State $n \gets \left\lceil \log_2\left(\frac{b_1 - a_1}{l}\right) \right\rceil$
        
        \For{$k = 1$ \textbf{to} $n$}
            \State $x \gets (a + b)/2$
            \State $df_m \gets f'(x)$
            
            \If{$df_m > 0$}
                \State $b \gets x$ \Comment{Το ελάχιστο βρίσκεται στο αριστερό μέρος}
            \ElsIf{$df_m < 0$}
                \State $a \gets x$ \Comment{Το ελάχιστο βρίσκεται στο δεξί μέρος}
            \Else
                \State $a \gets x$, $b \gets x$ \Comment{Βρέθηκε ακριβές ελάχιστο}
                \State \textbf{break}
            \EndIf
        \EndFor
        
        \State \Return $[a_K,b_K] \gets [a,b]$
        \end{algorithmic}
    \end{algorithm}
    
    \subsection{Γραφικές Παραστάσεις}

    Παρακάτω ακολουθούν οι γραφικές παραστάσεις της ανάλυσης που διεξαγάγαμε. 
    \begin{itemize}
        \item Στο \textit{Σχήμα~\ref{fig:task4_1}} παρουσιάζεται η μεταβολή των υπολογισμών της αντικειμενικής συνάρτησης $f_i$, για μεταβαλόμενη τελική ακρίβεια $l$.
        \item Στο \textit{Σχήμα~\ref{fig:task4_2}} παρουσιάζονται τα άκρα του διαστήματος $[a_k,b_k]$ συναρτήσει του δείκτη επαναλήψεων $k$ για διάφορα $l$.
    \end{itemize}

    \begin{figure}[H]
        \centering
        \includegraphics[width=0.75\linewidth]{assets/task4_1.jpg}
        \caption{Υπολογισμοί συνάρτησης για διαφορετικά $l$}
        \label{fig:task4_1}
    \end{figure}

    \begin{figure}[H]
        \centering
        \includegraphics[width=0.75\linewidth]{assets/task4_2.jpg}
        \caption{Άκρα του διαστήματος $[a_k,b_k]$ για διαφορετικά $k$ και $l$}
        \label{fig:task4_2}
    \end{figure}

    \vspace{1em}
    \subsection{Συμπεράσματα}

    \textbf{Ερώτημα 1:} Τα ευρήματα είναι παρόμοια με αυτά της Μεθόδου της Διχοτόμου.
    
    \textbf{Ερώτημα 2:} Τα ευρήματα είναι παρόμοια με αυτά της Μεθόδου της Διχοτόμου.

\section{Σύγκριση Μεθόδων}

    Σε αυτή την ενότητα θα χρησιμοποιήσουμε τα ευρήματα από τα όλα τα θέματα για να συγκρίνουμε τις μεθόδους μεταξύ τους ως προς την αποδοτικότητα.

    \begin{table}[H]
        \centering
        \small
        \renewcommand{\arraystretch}{1.12}
        \setlength{\tabcolsep}{6pt} % μικρότερη οριζόντια απόσταση κελιού
        \begin{tabularx}{\linewidth}{|p{0.17\linewidth}|Y|Y|Y|Y|}
            \hline
            \textbf{\makecell{Τελική \\ Ακρίβεια $l$}} &
            \textbf{Διχοτόμου} &
            \textbf{\makecell{Χρυσού \\ Τομέα}} &
            \textbf{Fibonacci} &
            \textbf{\makecell{Διχοτόμου \\ (παράγωγος)}} \\
            \hline
            0.01 & (10, 18) & (14, 15) & (14, 14) & (9, 9) \\
            \hline
            0.1  & (7, 12)  & (9, 10)  & (9, 9)   & (6, 6) \\
            \hline
        \end{tabularx}
        \caption{Επαναλήψεις $k$ και υπολογισμοί συνάρτησης ανά μέθοδο}
        \label{tab:algorithm-effectiveness}
    \end{table}

    \begin{itemize}
        \item \textbf{Παρατήρηση 1:} Με κριτήριο τους ελάχιστους υπολογισμούς συνάρτησης, η πιο αποδοτική μέθοδος φαίνεται να είναι η μέθοδος της διχοτόμου με χρήση παραγώγου. Επόμενη είναι η μέθοδος fibonacci, ακολουθούμενη από τη μέθοδο του χρυσού λόγου και τέλος την απλή μέθοδο της διχοτόμου.
    
        \item \textbf{Παρατήρηση 2:} Η μέθοδος Fibonacci περιλαμβάνει τον υπολογισμό μιας ακολουθίας Fibonacci μέχρι να ικανοποιηθεί η συνθήκη $F_n \ge (b_1-a_1)/l$. Αυτό σημαίνει πως πρέπει να προσδιοριστεί το $n$ πριν την εκτέλεση του κύριου βρόχου, και επομένως ο συνολικός αριθμός επαναλήψεων είναι προδιαγεγραμμένος. Στην πράξη, αυτό δίνει την ελάχιστη δυνατή (βέλτιστη) χρήση αξιολογήσεων για δεδομένο $l$, αλλά απαιτεί επιπλέον λογική για τον προσδιορισμό των όρων Fibonacci.
        
        \item \textbf{Παρατήρηση 3:} Η μέθοδος Fibonacci μοιάζει αρκετά με τη μέθοδο του Χρυσού Τομέα. Σε όρους λόγου μείωσης του μήκους του διαστήματος η Fibonacci προσεγγίζει (αριθμητικά) τη συμπεριφορά του χρυσού λόγου. Στην πράξη ο χρυσός λόγος $\gamma = (\sqrt5-1)/2 \approx 0.618$ προκύπτει ως όριο της σχέσης $\tfrac{F_{n-1}}{F_n}$ καθώς $n\to\infty$. Έτσι, για μεγάλα $n$ η μείωση του διαστήματος ανά επανάληψη συμπεριφέρεται όπως $(b_1-a_1)\gamma^n$.
        
        \item \textbf{Παρατήρηση 4:} Στη μέθοδο Διχοτόμου με χρήση παραγώγου, οι υπολογισμοί της αντικειμενικής συνάρτησης καθορίζονται από την παράγωγο στο μέσο κάθε βήματος. Εφόσον σε κάθε επανάληψη υπολογίζεται μόνο η παράγωγος στο ένα σημείο, ο αριθμός κλήσεων της \emph{συνάρτησης} μπορεί να είναι σαφώς μειωμένος σε σχέση με τις μη παραγώγους μεθόδους. Συγκεκριμένα, αν θεωρήσουμε ότι $k$ είναι ο αριθμός επαναλήψεων, τότε ο αριθμός των επιπλέον υπολογισμών που απαιτούνται για την παράγωγο είναι περίπου $n = k$ (ένας υπολογισμός παραγώγου ανά βήμα).
        
        \item \textbf{Παρατήρηση 5:} Ο συνολικός αριθμός αξιολογήσεων της $f$ μπορεί στην συγκεκριμένη άσκηση να μην επηρεάζει την ταχύτητα του αλγορίθμου με τον αναμενόμενο τρόπο όταν η συνάρτηση περιλαμβάνει απαιτητικές αριθμητικές πράξεις ή έντονες ταλαντώσεις. Σε αυτές τις περιπτώσεις το πραγματικό κόστος (χρόνος CPU, ακρίβεια αριθμητικών) και ο αριθμός κλήσεων της συνάρτησης λειτουργούν ως δύο διαφορετικοί παράγοντες: η μέθοδος με λιγότερες κλήσεις δεν είναι απαραίτητα ταχύτερη αν κάθε κλήση είναι πολύ ακριβή. Συνεπώς στην αξιολόγηση αποδοτικότητας πρέπει να λαμβάνεται υπόψη τόσο ο αριθμός κλήσεων όσο και το κόστος κάθε κλήσης.
    \end{itemize}

    Ακολουθεί πίνακας που συνοψίζει τα ευρήματα και τα συμπεράσματα της εργασίας αυτής:

    \begin{table}[H]
        \centering
        \small
        \begin{tabularx}{\linewidth}{|p{3cm}|X|X|p{3cm}|}
            \hline
            \textbf{Μέθοδος} &
            \textbf{Πλεονεκτήματα} &
            \textbf{Μειονεκτήματα} &
            \textbf{Χρήση} \\
            \hline
    
            Διχοτόμου &
            Απλότητα, εύκολη υλοποίηση, δεν απαιτεί παράγωγο. &
            Αργή σύγκλιση σε σχέση με χρυσό/Fibonacci, απαιτεί δύο αξιολογήσεις ανά βήμα. &
            Συνεχείς, μονότονες ή ομαλές συναρτήσεις, όταν θέλουμε απλή μέθοδο. \\
            \hline
    
            Χρυσού Τομέα &
            Απαιτεί ελάχιστες αξιολογήσεις (μετά τις αρχικές μία ανά επανάληψη), σταθερός ρυθμός μείωσης διαστήματος. &
            Ελαφρώς πιο πολύπλοκη υλοποίηση από τη διχοτόμου· απαιτεί σωστό υπολογισμό εσωτερικών σημείων. &
            Όταν θέλουμε αποδοτική μέθοδο χωρίς παράγωγο και χωρίς να γνωρίζουμε εκ των προτέρων τον αριθμό επαναλήψεων. \\
            \hline
    
            Fibonacci &
            Βέλτιστη ως προς αριθμό αξιολογήσεων για δεδομένη ακρίβεια (αν γνωρίζουμε εκ των προτέρων τα βήματα). &
            Απαιτεί προσδιορισμό της ακολουθίας Fibonacci και καθορισμό αριθμού επαναλήψεων εκ των προτέρων. &
            Όταν θέλουμε τον ελάχιστο αριθμό αξιολογήσεων και μπορούμε να προκαθορίσουμε τον αριθμό βημάτων. \\
            \hline
    
            Διχοτόμου με χρήση παραγώγου &
            Πολύ γρήγορη σύγκλιση όταν υπάρχει διαθέσιμη και αξιόπιστη παράγωγος. &
            Δεν εφαρμόζεται όταν η παράγωγος δεν είναι διαθέσιμη ή είναι ακριβή/θορυβώδης. &
            Όταν υπάρχει φθηνή και καλά ορισμένη παράγωγος και επιδιώκουμε ταχύτητα σύγκλισης. \\
            \hline
        \end{tabularx}
    
        \caption{Σύγκριση μεθόδων βελτιστοποίησης: πλεονεκτήματα, μειονεκτήματα και συνθήκες χρήσης.}
        \label{tab:method-comparison}
    \end{table}
    
\appendix

\section{Εργαλεία}
Τα εργαλεία που χρησιμοποιήθηκαν κατά την εκπόνηση της εργασίας είναι τα εξής:
\begin{itemize}
    \item MATLAB
    \item LaTeX
    \item Git
\end{itemize}

\section{Βιβλιογραφία}
\begin{thebibliography}{9}
    \bibitem[\href{https://www.tziola.gr/book/rovi/}{1}]{book}
         Γ. Ροβιθάκης, \textit{Τεχνικές Βελτιστοποίησης}, Εκδόσεις Τζιόλα, 2007.  
\end{thebibliography}

\end{document}