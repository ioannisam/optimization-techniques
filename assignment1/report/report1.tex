\documentclass[a4paper,12pt]{article}

\usepackage{fontspec}
\usepackage{polyglossia}

\setmainlanguage{greek}
\setotherlanguage{english}

\setmainfont{Latin Modern Roman}
\newfontfamily\greekfont{CMU Serif}[Script=Greek]

\usepackage[a4paper, top=1.5cm, bottom=1.5cm, left=1.5cm, right=1.5cm]{geometry}
\usepackage{graphicx}
\usepackage{float}
\usepackage{array}
\usepackage{hyperref}
\hypersetup{
    colorlinks=true,
    linkcolor=black,
    citecolor=black,
    urlcolor=blue
}
\usepackage[super,sort&compress]{cite}

\newcolumntype{C}{>{\centering\arraybackslash}c}

\title{Τεχνικές Βελτιστοποίησης: Εργασία 1}
\author{Ιωάννης Μιχάλαινας ΑΕΜ:10902}
\date{Νοέμβριος 2025}

\begin{document}

\maketitle

\begin{center}
\textbf{Code Repository:} \\
\href{https://github.com/ioannisam/optimization-techniques}
     {github.com/ioannisam/optimization-techniques}
\end{center}

\bigskip

\begin{abstract}
    Στην εργασία αυτή υλοποιούνται και συγκρίνονται μονοδιάστατες μέθοδοι βελτιστοποίησης - συγκεκριμένα η μέθοδος Διχοτόμου, η μέθοδος Χρυσού Τομέα, η μέθοδος Fibonacci και η Διχοτόμος Μέθοδος με χρήση παραγώγου. Για καθεμία από αυτές εξετάζεται ο αριθμός υπολογισμών και η ταχύτητα σύγκλισης, εφαρμόζοντάς τες σε διαφορετικές συναρτήσεις δοκιμής. Τα αποτελέσματα δείχνουν τις διαφορές αποδοτικότητας μεταξύ των μεθόδων και αναδεικνύουν τα πλεονεκτήματα και τους περιορισμούς καθεμιάς.
\end{abstract}

\tableofcontents
\clearpage

\section{Εισαγωγή}
Στην εργασία αυτή μελετάμε διάφορες τεχνικές βελτιστοποίησης, ονομαστικά τη μέθοδο Διχοτόμου, τη μέθοδο Χρυσού Τομέα, τη μέθοδο Fibonacci και τη μέθοδο Διχοτόμου με χρήση παραγώγου. Ο σκοπός της εργασίας αυτής είναι η εξοικείωση με τις παραπάνω μεθόδους και η σύγκριση μεταξύ τους, όσον αφορά την αποδοτικότητα, μέσω ανάλυσης του αριθμού υπολογισμών και της σύγκλισης των διαστημάτων αναζήτησης.

\section{Μέθοδος Διχοτόμου}

\section{Μέθοδος Χρυσού Τομέα}

\section{Μέθοδος Fibonacci}

\section{Μέθοδος Διχοτόμου (με χρήση παραγώγου)}

\appendix

\section{Εργαλεία}
Τα εργαλία που χρησιμοποιήθηκαν κατά την εκπόνηση της εργασίας είναι τα εξής:
\begin{itemize}
    \item MATLAB
    \item LaTeX
    \item Git
\end{itemize}

\section{Βιβλιογραφία}
\begin{thebibliography}{9}
    \bibitem[\href{https://www.tziola.gr/book/rovi/}{1}]{book}
         Γ. Ροβιθάκης, \textit{Τεχνικές Βελτιστοποίησης}, Εκδόσεις Τζιόλα, 2007.  
\end{thebibliography}

\end{document}