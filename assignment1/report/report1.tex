\documentclass[a4paper,12pt]{article}

\usepackage{fontspec}
\usepackage{polyglossia}

\setmainlanguage{greek}
\setotherlanguage{english}

\setmainfont{Latin Modern Roman}
\newfontfamily\greekfont{CMU Serif}[Script=Greek]

\usepackage[a4paper, top=1.5cm, bottom=1.5cm, left=1.5cm, right=1.5cm]{geometry}
\usepackage{graphicx}
\usepackage{float}
\usepackage{array}
\usepackage{hyperref}
\hypersetup{
    colorlinks=true,
    linkcolor=black,
    citecolor=black,
    urlcolor=blue
}
\usepackage[super,sort&compress]{cite}
\usepackage{algorithm}
\usepackage{algpseudocode}

\newcolumntype{C}{>{\centering\arraybackslash}c}

\title{Τεχνικές Βελτιστοποίησης: Εργασία 1}
\author{Ιωάννης Μιχάλαινας ΑΕΜ:10902}
\date{Νοέμβριος 2025}

\begin{document}

\maketitle

\begin{center}
\textbf{Αποθετήριο Κώδικα:} \\
\href{https://github.com/ioannisam/optimization-techniques}
     {github.com/ioannisam/optimization-techniques}
\end{center}

\bigskip

\begin{abstract}
    Στην εργασία αυτή υλοποιούνται και συγκρίνονται μονοδιάστατες μέθοδοι βελτιστοποίησης - συγκεκριμένα η μέθοδος Διχοτόμου, η μέθοδος Χρυσού Τομέα, η μέθοδος Fibonacci και η Διχοτόμος Μέθοδος με χρήση παραγώγου. Για καθεμία από αυτές εξετάζεται ο αριθμός υπολογισμών και η ταχύτητα σύγκλισης, εφαρμόζοντάς τες σε διαφορετικές συναρτήσεις δοκιμής. Τα αποτελέσματα δείχνουν τις διαφορές αποδοτικότητας μεταξύ των μεθόδων και αναδεικνύουν τα πλεονεκτήματα και τους περιορισμούς καθεμιάς.
\end{abstract}

\tableofcontents
\clearpage

\section{Εισαγωγή}
Στην εργασία αυτή μελετάμε διάφορες τεχνικές βελτιστοποίησης, ονομαστικά τη μέθοδο Διχοτόμου, τη μέθοδο Χρυσού Τομέα, τη μέθοδο Fibonacci και τη μέθοδο Διχοτόμου με χρήση παραγώγου. Ο σκοπός της εργασίας αυτής είναι η εξοικείωση με τις παραπάνω μεθόδους και η σύγκριση μεταξύ τους, όσον αφορά την αποδοτικότητα, μέσω ανάλυσης του αριθμού υπολογισμών και της σύγκλισης των διαστημάτων αναζήτησης.

\section{Μέθοδος Διχοτόμου}

    \subsection{Αλγόριθμος}
    
    \begin{algorithm}
        \caption{Μέθοδος Διχοτόμου}
        \begin{algorithmic}[1]
        \Require συνάρτηση $f$, διάστημα $[a_1,b_1]$, σταθερά $\epsilon>0$, τελική ακρίβεια $l>0$
        \Ensure προσέγγιση ελαχίστου $x^*$
        \State ορίστε $a \gets a_1$, $b \gets b_1$, $k \gets 1$
        \While{$b - a > l$}
            \State $m \gets (a+b)/2$
            \State $x_1 \gets m - \epsilon$
            \State $x_2 \gets m + \epsilon$
            \State $f_1 \gets f(x_1)$
            \State $f_2 \gets f(x_2)$
            \If{$f_1 < f_2$}
                \State $b \gets x_2$ \Comment{Το ελάχιστο βρίσκεται στο [a, x2]}
            \Else
                \State $a \gets x_1$ \Comment{Το ελάχιστο βρίσκεται στο [x1, b]}
            \EndIf
            \State $k \gets k+1$
        \EndWhile
        
        \State \Return $[a_K,b_K] \gets [a,b]$
        \end{algorithmic}
    \end{algorithm}
    
    \subsection{Γραφικές Παραστάσεις}
    \subsection{Συμπεράσματα}
\section{Μέθοδος Χρυσού Τομέα}

    \subsection{Αλγόριθμος}
    
    \begin{algorithm}
        \caption{Μέθοδος Χρυσού Τομέα}
        \begin{algorithmic}[1]
        \Require συνάρτηση $f$, διάστημα $[a_1,b_1]$, τελική ακρίβεια $l>0$
        \Ensure προσέγγιση ελαχίστου $x^*$
        \State ορίστε $a \gets a_1$, $b \gets b_1$, $k \gets 1$
        \State $\gamma \gets (\sqrt{5}-1)/2$ \Comment{Χρυσός λόγος (0.618)}
        \State $x_1 \gets b - (1-\gamma)*(b-a)$
        \State $x_2 \gets a + \gamma*(b-a)$
        \State $f_1 \gets f(x_1)$
        \State $f_2 \gets f(x_2)$
        \While{$b - a > l$}
            \If{$f_1 > f_2$}
                \State $a \gets x_1$
                \State $x_1 \gets x_2$
                \State $f_1 \gets f_2$
                \State $x_2 \gets a + \gamma*(b-a)$
                \State $f_2 \gets f(x_2)$
            \Else
                \State $b \gets x_2$
                \State $x_2 \gets x_1$
                \State $f_2 \gets f_1$
                \State $x_1 \gets b - (1-\gamma)*(b-a)$
                \State $f_1 \gets f(x_1)$ 
            \EndIf
            \State $k \gets k + 1$
        \EndWhile
        
        \State \Return $[a_K,b_K] \gets [a,b]$
        \end{algorithmic}
    \end{algorithm}
    
    \subsection{Γραφικές Παραστάσεις}
    \subsection{Συμπεράσματα}
\section{Μέθοδος Fibonacci}

    \subsection{Αλγόριθμος}
    
    \begin{algorithm}
        \caption{Μέθοδος Fibonacci}
        \begin{algorithmic}[1]
        \Require συνάρτηση $f$, διάστημα $[a_1,b_1]$, σταθερά $\epsilon>0$, τελική ακρίβεια $l>0$
        \Ensure προσέγγιση ελαχίστου $x^*$
        \State Ορίστε $a \gets a_1$, $b \gets b_1$, $k \gets 1$
        \State Δημιουργήστε τη σειρά Fibonacci $F = [1, 1]$
        \While{$F_{\text{end}} < (b-a)/l$}
            \State Προσθέστε $F_{\text{end}} + F_{\text{end-1}}$ στο τέλος της σειράς
        \EndWhile
        \State $n \gets \text{μήκος}(F)$
        \State $x_1 \gets a + (F_{n-2}/F_n)*(b-a)$
        \State $x_2 \gets a + (F_{n-1}/F_n)*(b-a)$
        \State $f_1 \gets f(x_1)$
        \State $f_2 \gets f(x_2)$
        
        \For{$k = 1$ to $n-3$}
            \If{$f_1 > f_2$}
                \State $a \gets x_1$
                \State $x_1 \gets x_2$
                \State $f_1 \gets f_2$
                \State $x_2 \gets a + (F_{n-k-1}/F_{n-k})*(b-a)$
                \State $f_2 \gets f(x_2)$
            \Else
                \State $b \gets x_2$
                \State $x_2 \gets x_1$
                \State $f_2 \gets f_1$
                \State $x_1 \gets a + (F_{n-k-2}/F_{n-k})*(b-a)$
                \State $f_1 \gets f(x_1)$
            \EndIf
        \EndFor

        \State $x_2 \gets x_1 + \epsilon$ 
        \If{$f_1 > f_2$}
            \State $a \gets x_1$
        \Else
            \State $b \gets x_2$
        \EndIf
        
        \State \Return $[a_K,b_K] \gets [a,b]$
        \end{algorithmic}
    \end{algorithm}
    
    \subsection{Γραφικές Παραστάσεις}
    \subsection{Συμπεράσματα}
\section{Μέθοδος Διχοτόμου (με χρήση παραγώγου)}

    \subsection{Αλγόριθμος}

    \begin{algorithm}
    \caption{Μέθοδος Διχοτόμου (με χρήση παραγώγου)}
    \begin{algorithmic}[1]
        \Require παράγωγος συνάρτησης $f'$, διάστημα $[a_1,b_1]$, τελική ακρίβεια $l>0$
        \Ensure τελικό διάστημα $[a_K,b_K]$ που περιέχει το ελάχιστο
    
        \State Ορίστε $a \gets a_1$, $b \gets b_1$
        \State $n \gets \left\lceil \log_2\left(\frac{b_1 - a_1}{l}\right) \right\rceil$
        
        \For{$k = 1$ \textbf{to} $n$}
            \State $x \gets (a + b)/2$
            \State $df_m \gets f'(x)$
            
            \If{$df_m > 0$}
                \State $b \gets x$ \Comment{Το ελάχιστο βρίσκεται στο αριστερό μέρος}
            \ElsIf{$df_m < 0$}
                \State $a \gets x$ \Comment{Το ελάχιστο βρίσκεται στο δεξί μέρος}
            \Else
                \State $a \gets x$, $b \gets x$ \Comment{Βρέθηκε ακριβές ελάχιστο}
                \State \textbf{break}
            \EndIf
        \EndFor
        
        \State \Return $[a_K,b_K] \gets [a,b]$
        \end{algorithmic}
    \end{algorithm}
    
    \subsection{Γραφικές Παραστάσεις}
    \subsection{Συμπεράσματα}
\section{Σύγκριση Μεθόδων}

\appendix

\section{Εργαλεία}
Τα εργαλία που χρησιμοποιήθηκαν κατά την εκπόνηση της εργασίας είναι τα εξής:
\begin{itemize}
    \item MATLAB
    \item LaTeX
    \item Git
\end{itemize}

\section{Βιβλιογραφία}
\begin{thebibliography}{9}
    \bibitem[\href{https://www.tziola.gr/book/rovi/}{1}]{book}
         Γ. Ροβιθάκης, \textit{Τεχνικές Βελτιστοποίησης}, Εκδόσεις Τζιόλα, 2007.  
\end{thebibliography}

\end{document}